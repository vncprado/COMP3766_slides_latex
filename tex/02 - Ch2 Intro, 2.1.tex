% Beamer presentation with Memorial University theme
\documentclass[t]{beamer}
% Use custom Memorial University theme (explicit input to ensure findability)
% Minimal Memorial beamer theme: only red/grey/black/white
% Keeps styles minimal to produce a clean title page

\mode<presentation>

% Colors
\definecolor{MemorialRed}{RGB}{134,38,51}
\definecolor{MemorialGrey}{RGB}{99,102,106}
\definecolor{MemorialWhite}{RGB}{255,255,255}
\definecolor{MemorialBlack}{RGB}{35,31,32}

% Palette
\setbeamercolor{palette primary}{bg=MemorialRed, fg=MemorialWhite}
\setbeamercolor{palette secondary}{bg=MemorialGrey, fg=MemorialWhite}
\setbeamercolor{structure}{fg=MemorialRed}
\setbeamercolor{title}{fg=MemorialWhite, bg=MemorialRed}
\setbeamercolor{frametitle}{fg=MemorialWhite, bg=MemorialRed}
\setbeamercolor{normal text}{fg=MemorialBlack, bg=MemorialWhite}

% Fonts
\usefonttheme{professionalfonts}
\setbeamerfont{title}{size=\Large, series=\bfseries}
\setbeamerfont{frametitle}{size=\large, series=\bfseries}

\setbeamersize{text margin left=0.5cm, text margin right=0.5cm}

% Remove navigation symbols
\setbeamertemplate{navigation symbols}{}

% Headline with section name on the right
\setbeamertemplate{headline}{%
  \begin{beamercolorbox}[wd=1\paperwidth, ht=0.45cm, dp=0pt]{palette secondary}
    \vbox to 0.45cm{\vfil%
      \hbox to 1\paperwidth{%
        \hspace{1em}\hfill\usebeamerfont{section in head/foot}\insertsectionhead\hspace{1em}%
      }%
      \vfil}
  \end{beamercolorbox}
}

% Simple footline: muted background, left author, right page numbers
\setbeamertemplate{footline}{%
  \begin{beamercolorbox}[wd=1\paperwidth, ht=0.45cm, dp=0pt]{palette secondary}
    \vbox to 0.45cm{\vfil%
      \hbox to 1\paperwidth{%
        \hspace{1em}\usebeamerfont{author in head/foot}\insertshortauthor\hfill\insertframenumber/\inserttotalframenumber\hspace{1em}%
      }%
      \vfil}
  \end{beamercolorbox}
}

% Boxed title page (restores a visually framed title like the beamer default)
\setbeamertemplate{title page}{%
  \begin{center}
    \vspace{1cm}
    \begin{beamercolorbox}[sep=1.25em,center,rounded=true,shadow=true]{title}
      {\usebeamerfont{title}\inserttitle}\par\vskip0.5em
      {\usebeamerfont{subtitle}\insertsubtitle}
    \end{beamercolorbox}
    \vskip1.2em
    {\usebeamerfont{author}\insertauthor}\par\vskip0.5em
    {\usebeamerfont{date}\insertdate}
    \vfill
  \end{center}
}

\mode<all>



\usepackage{soul}
\usepackage{graphicx}
\usepackage{subcaption}
\usepackage{hyperref}


% Define email
\def\email{vpradodafons@mun.ca}

\title{02 - Ch2 Intro, 2.1, 2.2}
% short author (used in footer) set to email
\author{Vinicius Prado da Fonseca, Ph.D. \\ vpradodafons@mun.ca}
\date{\today}

\begin{document}

\begin{frame}

  \titlepage

\end{frame}

\begin{frame}{Outline}

  \begin{columns}[T]
        \begin{column}{\textwidth}
          \tableofcontents
        \end{column}
  \end{columns}

\end{frame}

\section{Ch 2 Intro}

\begin{frame}

  A robot is mechanically constructed by connecting \textbf{rigid bodies}, called \textbf{links}, together by means of \textbf{joints}, so that relative motion between adjacent links becomes possible.
  \textbf{Actuation} of the joints, typically by electric motors, then causes the robot to move and exert forces in desired ways.

  \begin{center}
    \includegraphics[height=0.6\paperheight]{imgs/02/fig2-14.png}     
  \end{center}

\end{frame}

\section{Where is the robot?}
\begin{frame}{Where is the robot?}

  The answer is given by the robot’s \textbf{configuration}: a specification of the positions of all points of the robot.
  Since the robot’s links are rigid and of a known shape, only a few numbers are needed to represent its configuration.
  Imagine a soft material like a pillow. You need to know where are all the points of an object to describe it.
  With rigid mechanisms only a few numbers are needed to represent its configuration.
  What is the the configuration of a door? (Fig 2.1 (a)).

\end{frame}

\begin{frame}

  The configuration of a point on a plane can be described by two coordinates, $(x, y)$.
  The configuration of a coin lying heads up on a flat table can be described by three coordinates:
  \begin{itemize}
    \item two coordinates $(x, y)$ that specify the location of a particular point on the coin, and
    \item one coordinate $(\theta)$ that specifies the coin’s orientation.
  \end{itemize}

\end{frame}


\begin{frame}
  The coordinates below all take values over a continuous range of real numbers.

  \begin{center}
    \includegraphics[height=0.6\paperheight]{imgs/02/fig2-1.png} 
  \end{center}

\end{frame}

\begin{frame}

  The number of \textbf{degrees of freedom (dof)} of a robot is the smallest number of real-valued coordinates needed to represent its configuration.
  
  In the example above,

\begin{itemize}
  \item the door has one degree of freedom, and
  \item the coin lying heads up on a table has three degrees of freedom.
\end{itemize}

Even if the coin could lie either heads up or tails up, its configuration space still would have only three degrees of freedom;

a fourth variable, representing which side of the coin faces up, takes values in the discrete set {heads,tails}, and not over a continuous range of real values like the other three coordinates.

\end{frame}

\begin{frame}[c]

  \begin{center}
    \includegraphics[width=\textwidth]{imgs/02/def2-1.png}
  \end{center}

\end{frame}

\begin{frame}

  In this chapter we study the C-space and degrees of freedom of general robots.
  Since our robots are constructed from rigid links, we examine first the degrees of freedom of a single rigid body, and then the degrees of freedom of general multi-link robots.
  Next we study the shape (or topology) and geometry of C-spaces and their mathematical representation.
  The chapter concludes with a discussion of the C-space of a robot’s end-effector, its task space.
  In the following chapter we study in more detail the mathematical representation of the C-space of a single rigid body.

\end{frame}

\section{2.1. Degrees of Freedom of a Rigid Body}
\begin{frame}{2.1. Degrees of Freedom of a Rigid Body}

  \begin{columns}[T]
    % --- Column 1: Text ---
    \begin{column}{0.4\textwidth}
      Continuing with the example of the coin lying on the table, choose three points A, B, and C on the coin. Once a coordinate frame $\hat{x}-\hat{y}$ is attached to the plane, the positions of these points in the plane are written $(x_A, y_A)$, $(x_B, y_B)$, and $(x_C , y_C )$.
    \end{column}

    % --- Column 2: Figure ---
    \begin{column}{0.6\textwidth} % 50% of the text width
        \centering % Center the image within this column
        \includegraphics[width=0.6\textwidth]{imgs/02/fig2-2a.png} 
        % \captionof{figure}{A description of the image.}
    \end{column}
  \end{columns}
\end{frame}

\begin{frame}

  But, according to the definition of a rigid body, the distance between point A and point B, denoted $d(A, B)$, is always constant regardless of where the coin is.
Similarly, the distances $d(B, C)$ and $d(A, C)$ must be constant.

The following equality constraints on the coordinates $(x_A, y_A)$, $(x_B, y_B)$, and $(x_C , y_C)$ must therefore always be satisfied:

\begin{align*}
  d(A, B) &= \sqrt{(x_A - x_B)^2 + (y_A - y_B)^2} = d_{AB}, \\
  d(B, C) &= \sqrt{(x_B - x_C)^2 + (y_B - y_C)^2} = d_{BC}, \\
  d(A, C) &= \sqrt{(x_A - x_C)^2 + (y_A - y_C)^2} = d_{AC}.
\end{align*}
  
\end{frame}

\begin{frame}

To determine the number of degrees of freedom of the coin on the table,

\begin{enumerate}
  \item Choose the position of point A in the plane (Figure 2.2(b)).
  \item We may choose it to be anything we want, so we have two degrees of freedom to specify, namely $(x_A, y_A)$.
  \item Once $(x_A, y_A)$ is specified, the constraint $d(A, B) = d_{AB}$ restricts the choice of $(x_B, y_B)$;
  \item We must choose those points on the circle of radius $d_{AB}$ centered at A.
  \item A point on this circle can be specified by a single parameter, e.g., the angle specifying the location of B on the circle centered at A.
  \item Let’s call this angle $\varphi_{AB}$ and define it to be the angle that the vector $\overrightarrow{AB}$ makes with the $\hat{x}$-axis.

\end{enumerate}

\end{frame}

\begin{frame}

  Once we have chosen the location of point B, there are only two possible locations for C:
  \begin{itemize}
    \item  at the intersections of the circle of radius $d_{AC}$ centered at A
    \item and the circle of radius $d_{BC}$ centered at B (Figure 2.2(b)). 
  \end{itemize}
  These two solutions correspond to heads or tails.
  In other words, once we have placed A and B and chosen heads or tails, the two constraints $d(A, C) = d_{AC}$ and $d(B, C) = d_{BC}$ eliminate the two apparent freedoms provided by (xC , yC ), and the location of C is fixed.
  The coin has exactly three degrees of freedom in the plane, which can be specified by $(x_A, y_A, \varphi_{AB})$.

\end{frame}

\begin{frame}

  Suppose that we choose to specify the position of an additional point D on the coin.
  This introduces three additional constraints: $d(A, D) = d_{AD}$, $d(B, D) = d_{BD}$, and $d(C, D) = d_{CD}$.
  One of these constraints is redundant, i.e., it provides no new information; only two of the three constraints are independent. 

\end{frame}

\begin{frame}
  We have been applying the following general rule for determining the number of degrees of freedom of a system:

  \begin{center}
    \includegraphics[width=0.9\textwidth]{imgs/02/eq2-1.png} 
  \end{center}

  This rule can also be expressed in terms of the number of variables and independent equations that describe the system:

  \begin{center}
    \includegraphics[width=0.9\textwidth]{imgs/02/eq2-2.png} 
  \end{center}  
\end{frame}

\begin{frame}
  
  This general rule can also be used to determine the number of freedoms of a rigid body in three dimensions.
  The coordinates for the three points A, B, and C are now given by $(x_A, y_A, z_A)$, $(x_B, y_B, z_B)$, and $(x_C , y_C , z_C )$, respectively.

  \begin{itemize}
    \item Point A can be placed freely (three degrees of freedom).
    \item The location of point B is subject to the constraint $d(A, B) = d_{AB}$, meaning it must lie on the sphere of radius $d_{AB}$ centered at A.
    \item Thus we have $3-1 = 2$ freedoms to specify, which can be expressed as the latitude and longitude for the point on the sphere.
  \end{itemize}

\end{frame}

\begin{frame}
  
  \begin{itemize}
    \item Finally, the location of point C must lie at the intersection of spheres centered at A and B of radius $d_{AC}$ and $d_{BC}$, respectively.
    \item In the general case the intersection of two spheres is a circle, and the location of point C can be described by an angle that parametrizes this circle.
    \item Point C therefore adds $3-2 = 1$ freedom.
    \item Once the position of point C is chosen, the coin is fixed in space.
  \end{itemize}

  In summary, a rigid body in three-dimensional space has six freedoms, which
can be described by the three coordinates parametrizing point A, the two angles
parametrizing point B, and one angle parametrizing point C, provided A, B,
and C are noncollinear. Other representations for the configuration of a rigid
body are discussed in Chapter 3.
\end{frame}


\begin{frame}

We have just established that a rigid body moving in three-dimensional
space, which we call a spatial rigid body, has six degrees of freedom. Similarly,
a rigid body moving in a two-dimensional plane, which we henceforth call a
planar rigid body, has three degrees of freedom. This latter result can also
be obtained by considering the planar rigid body to be a spatial rigid body with
six degrees of freedom but with the three independent constraints $z_A = z_B =
z_C = 0$.
Since our robots consist of rigid bodies, Equation (2.1) can be expressed as
follows:
  \begin{center}
    \includegraphics[width=0.9\textwidth]{imgs/02/eq2-3.png} 
  \end{center}  
Equation (2.3) forms the basis for determining the degrees of freedom of general
robots, which is the topic of the next section.
\end{frame}

\section*{} % Final slide use an unnumbered section
\begin{frame}{Questions?}

  Next:
  \begin{itemize}
    \item 03
  \end{itemize}

\end{frame}

\end{document}
