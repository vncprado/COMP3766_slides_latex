% Beamer presentation with Memorial University theme
\documentclass{beamer}
% Use custom Memorial University theme (explicit input to ensure findability)
% Minimal Memorial beamer theme: only red/grey/black/white
% Keeps styles minimal to produce a clean title page

\mode<presentation>

% Colors
\definecolor{MemorialRed}{RGB}{134,38,51}
\definecolor{MemorialGrey}{RGB}{99,102,106}
\definecolor{MemorialWhite}{RGB}{255,255,255}
\definecolor{MemorialBlack}{RGB}{35,31,32}

% Palette
\setbeamercolor{palette primary}{bg=MemorialRed, fg=MemorialWhite}
\setbeamercolor{palette secondary}{bg=MemorialGrey, fg=MemorialWhite}
\setbeamercolor{structure}{fg=MemorialRed}
\setbeamercolor{title}{fg=MemorialWhite, bg=MemorialRed}
\setbeamercolor{frametitle}{fg=MemorialWhite, bg=MemorialRed}
\setbeamercolor{normal text}{fg=MemorialBlack, bg=MemorialWhite}

% Fonts
\usefonttheme{professionalfonts}
\setbeamerfont{title}{size=\Large, series=\bfseries}
\setbeamerfont{frametitle}{size=\large, series=\bfseries}

\setbeamersize{text margin left=0.5cm, text margin right=0.5cm}

% Remove navigation symbols
\setbeamertemplate{navigation symbols}{}

% Headline with section name on the right
\setbeamertemplate{headline}{%
  \begin{beamercolorbox}[wd=1\paperwidth, ht=0.45cm, dp=0pt]{palette secondary}
    \vbox to 0.45cm{\vfil%
      \hbox to 1\paperwidth{%
        \hspace{1em}\hfill\usebeamerfont{section in head/foot}\insertsectionhead\hspace{1em}%
      }%
      \vfil}
  \end{beamercolorbox}
}

% Simple footline: muted background, left author, right page numbers
\setbeamertemplate{footline}{%
  \begin{beamercolorbox}[wd=1\paperwidth, ht=0.45cm, dp=0pt]{palette secondary}
    \vbox to 0.45cm{\vfil%
      \hbox to 1\paperwidth{%
        \hspace{1em}\usebeamerfont{author in head/foot}\insertshortauthor\hfill\insertframenumber/\inserttotalframenumber\hspace{1em}%
      }%
      \vfil}
  \end{beamercolorbox}
}

% Boxed title page (restores a visually framed title like the beamer default)
\setbeamertemplate{title page}{%
  \begin{center}
    \vspace{1cm}
    \begin{beamercolorbox}[sep=1.25em,center,rounded=true,shadow=true]{title}
      {\usebeamerfont{title}\inserttitle}\par\vskip0.5em
      {\usebeamerfont{subtitle}\insertsubtitle}
    \end{beamercolorbox}
    \vskip1.2em
    {\usebeamerfont{author}\insertauthor}\par\vskip0.5em
    {\usebeamerfont{date}\insertdate}
    \vfill
  \end{center}
}

\mode<all>



\usepackage{soul}
\usepackage{graphicx}
\usepackage{subcaption}
\usepackage{hyperref}


% Define email
\def\email{vpradodafons@mun.ca}

\title{01 - Outline, Ch1 Preview}
% short author (used in footer) set to email
\author{Vinicius Prado da Fonseca, Ph.D. \\ vpradodafons@mun.ca}
\date{\today}

\begin{document}

\begin{frame}
  \titlepage
\end{frame}

\begin{frame}{Outline}
  \tableofcontents
  % \begin{itemize}
  %   \item Introduction
  %   \item Example slide
  %   \item Conclusion
  % \end{itemize}
\end{frame}

\section{Course overview}
\begin{frame}{Course objectives}
  \begin{itemize}
    \item The primary objective: introduction to the mathematical formulation and practical aspects of robotic manipulators.
    \item It will present kinematics, dynamics, control, and programming vital to effectively using/designing  robotic arms.
    \item Topics covered:
    \begin{itemize}
      \item Configuration space;
      \item Rigid-body motions;
      \item Forward and inverse kinematics;
      \item Trajectory generation;
      \item Motion planning;
      \item Manipulator control.
    \end{itemize}
    \item A complete yet straightforward robotic manipulator model will be developed to demonstrate these concepts using high-level languages and frameworks.
  \end{itemize}
\end{frame}

\begin{frame}{Course details}
  \begin{itemize}
    \item Instructor: Dr. Vinicius Prado da Fonseca
    \begin{itemize}
      \item E-mail: \textbf{vpradodafons@online.mun.ca (Brightspace email) include [COMP3766] in the subject line};
        \begin{itemize}
          \item Send email within Brightspace. Does not receive emails from outside brightspace;
          \item Use ONLY brightspace email.
        \end{itemize}
    \item Office: EN-2012;
    \item Office Hours: Friday after class or by appointment.
    \end{itemize}
    \item Course Prerequisites: COMP2001, COMP2002, MATH2000, MATH2050, STAT2500 or STAT2550
    \item Lectures (\textbf{EN 1054, M-W-F 14:00 - 14:50}) and course notes on Brightspace;
    \begin{itemize}
      \item Lectures are recorded on Online Rooms (Bongo).
    \end{itemize}
  \end{itemize}
\end{frame}

\begin{frame}{Course details (cont.)}
  \begin{itemize}
    \item Labs
    \begin{itemize}
      \item Lecture room, lecture time, some Fridays;
      \item Quiz on Brightspace;
      \item Course repository: \url{https://github.com/vncprado/COMP3766}.
      \item More details next slides.
    \end{itemize}
    \item Textbook
      \begin{itemize}
        \item Modern Robotics Mechanics, Planning, And Control, Kevin M. Lynch and Frank C. Park, 2019 (Cambridge University Press);
        \item Book website: \url{https://hades.mech.northwestern.edu/index.php/Modern_Robotics}
        \item Online notes and slides will also be available.
      \end{itemize}
  \end{itemize}
\end{frame}

\begin{frame}{Evaluation}
  \begin{itemize}
    \item The final grade in the course will be determined as follows:
    \begin{itemize}
      \item   Quizzes (2) - 25\%;
      \item   Lab exercises (4) - 20\%;
      \item   Assignments (\hl{5 or 6}) - 30\%;
      \item   Final Project - 25\%.
    \end{itemize}
    \item Course outline on Brightspace:
      \begin{itemize}
        \item Evaluation, lecture dates, lab schedule, assignment deadlines and textbook units.
      \end{itemize}
  \end{itemize}
\end{frame}

\begin{frame}{Assignments}
  \begin{itemize}
    \item Assignments (\hl{5 or 6}) may have written and/or programming portions
    \item Submit written portions by hand in a scanned document or annotated PDF.
    \item Programming portions will be mostly in Python. Use the provided files as a template for your code.  
  \end{itemize}
\end{frame}

\begin{frame}{Labs}
  \begin{itemize}
    \item Labs (4) will be \textbf{IN PERSON}
    \begin{itemize}
      \item 
      \item Lecture room, Lecture time, some Fridays, see schedule on Brightspace;
      \item Repository with ROS workspace will be provided;
      \item Bring laptop with \textbf{Docker} and \textbf{VSCode} installed;
      \item D2L Quiz will be open until 23:59 on lab day.
    \end{itemize}
    \item NO EXTENSIONS!
    \item You must do the \ul{\textbf{pre-lab section}} before arriving at the lab time.
    \item After completing the lab you must answer the \textbf{quiz on brightspace}.
    \item On Brightspace you will find:
    \begin{itemize}
      \item Lab schedule; 
      \item Lab manual;
      \item Support files;
      \item Lab quiz.
    \end{itemize}
  \end{itemize}  
\end{frame}

\begin{frame}{Final Project}
  \begin{itemize}
    \item Repository (\hl{INCLUDE REPO LINK HERE})
    \item Final weeks starting after second quiz.
    \item Deadline last day of classes.
  \end{itemize}
\end{frame}

\section{Ch1 Preview}
\begin{frame}
  Robotics is a relatively young discipline with highly ambitious goals, the ultimate one being the creation of machines that can behave and think like humans.  

  \begin{figure}[h]
      \centering
      % TODO: UPDATE THIS LINK TO THE REPO GIFS
      \href{https://www.google.com}{
          \includegraphics[width=0.5\textwidth]{imgs/01/atlas-yellow.jpg}
      }
      \caption{Boston Dynamics Atlas Robot.}
      \label{fig:atlas-clickable}
  \end{figure}

%   \begin{figure}[htb!]
%     \centering
%     % Define a fixed width for all three subfigures.
%     % In this case, each image is slightly less than 1/3 of the text width 
%     % to account for spacing between them.
%     \begin{subfigure}[b]{0.3\textwidth}
%         \centering
%         \includegraphics[width=\textwidth]{imgs/01/spot.gif} 
%         \caption{Spot robot.}
%         \label{fig:gif1}
%     \end{subfigure}% % <--- Crucial to prevent extra space (like a newline)
%     \hfill% % <--- Adds flexible horizontal space between subfigures
%     \begin{subfigure}[b]{0.3\textwidth}
%         \centering
%         \includegraphics[width=\textwidth]{imgs/01/atlas_fail.gif}
%         \caption{Atlas falling while jumping.}
%         \label{fig:gif2}
%     \end{subfigure}%
%     \hfill%
%     \begin{subfigure}[b]{0.3\textwidth}
%         \centering
%         \includegraphics[width=\textwidth]{imgs/01/atlas_nofail.gif}
%         \caption{Atlas jumping successfully.}
%         \label{fig:gif3}
%     \end{subfigure}
    
%     \caption{Three robots side-by-side.}
%     \label{fig:three_gifs}
% \end{figure}
\end{frame}

\end{document}
