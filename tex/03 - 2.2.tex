% Beamer presentation with Memorial University theme
\documentclass[t]{beamer}
% Use custom Memorial University theme (explicit input to ensure findability)
% Minimal Memorial beamer theme: only red/grey/black/white
% Keeps styles minimal to produce a clean title page

\mode<presentation>

% Colors
\definecolor{MemorialRed}{RGB}{134,38,51}
\definecolor{MemorialGrey}{RGB}{99,102,106}
\definecolor{MemorialWhite}{RGB}{255,255,255}
\definecolor{MemorialBlack}{RGB}{35,31,32}

% Palette
\setbeamercolor{palette primary}{bg=MemorialRed, fg=MemorialWhite}
\setbeamercolor{palette secondary}{bg=MemorialGrey, fg=MemorialWhite}
\setbeamercolor{structure}{fg=MemorialRed}
\setbeamercolor{title}{fg=MemorialWhite, bg=MemorialRed}
\setbeamercolor{frametitle}{fg=MemorialWhite, bg=MemorialRed}
\setbeamercolor{normal text}{fg=MemorialBlack, bg=MemorialWhite}

% Fonts
\usefonttheme{professionalfonts}
\setbeamerfont{title}{size=\Large, series=\bfseries}
\setbeamerfont{frametitle}{size=\large, series=\bfseries}

\setbeamersize{text margin left=0.5cm, text margin right=0.5cm}

% Remove navigation symbols
\setbeamertemplate{navigation symbols}{}

% Headline with section name on the right
\setbeamertemplate{headline}{%
  \begin{beamercolorbox}[wd=1\paperwidth, ht=0.45cm, dp=0pt]{palette secondary}
    \vbox to 0.45cm{\vfil%
      \hbox to 1\paperwidth{%
        \hspace{1em}\hfill\usebeamerfont{section in head/foot}\insertsectionhead\hspace{1em}%
      }%
      \vfil}
  \end{beamercolorbox}
}

% Simple footline: muted background, left author, right page numbers
\setbeamertemplate{footline}{%
  \begin{beamercolorbox}[wd=1\paperwidth, ht=0.45cm, dp=0pt]{palette secondary}
    \vbox to 0.45cm{\vfil%
      \hbox to 1\paperwidth{%
        \hspace{1em}\usebeamerfont{author in head/foot}\insertshortauthor\hfill\insertframenumber/\inserttotalframenumber\hspace{1em}%
      }%
      \vfil}
  \end{beamercolorbox}
}

% Boxed title page (restores a visually framed title like the beamer default)
\setbeamertemplate{title page}{%
  \begin{center}
    \vspace{1cm}
    \begin{beamercolorbox}[sep=1.25em,center,rounded=true,shadow=true]{title}
      {\usebeamerfont{title}\inserttitle}\par\vskip0.5em
      {\usebeamerfont{subtitle}\insertsubtitle}
    \end{beamercolorbox}
    \vskip1.2em
    {\usebeamerfont{author}\insertauthor}\par\vskip0.5em
    {\usebeamerfont{date}\insertdate}
    \vfill
  \end{center}
}

\mode<all>



\usepackage{soul}
\usepackage{graphicx}
\usepackage{subcaption}
\usepackage{hyperref}


% Define email
\def\email{vpradodafons@mun.ca}

\title{03 - 2.2}
% short author (used in footer) set to email
\author{Vinicius Prado da Fonseca, Ph.D. \\ vpradodafons@mun.ca}
\date{\today}

\begin{document}

\begin{frame}

  \titlepage

\end{frame}

\begin{frame}{Outline}

  \begin{columns}[T]
        \begin{column}{\textwidth}
          \tableofcontents
        \end{column}
  \end{columns}

\end{frame}

\section{2.2 Degrees of Freedom of a Robot}
\begin{frame}{2.2 Degrees of Freedom of a Robot}

  \begin{columns}[c]
    % --- Column left
    \begin{column}{0.6\textwidth}
      Consider once again the door example of Figure 2.1(a)
      \begin{itemize}
        \item It consists of a single rigid body connected to a wall by a hinge joint.
        \item It has only one degree of freedom, conveniently represented by the hinge joint angle $\theta$.
        \item Without the hinge joint, the door would have six degrees of freedom.
        \item By connecting the door to the wall via the hinge joint, five independent constraints are imposed, one independent coordinate ($\theta$). 
        \item The door's C-space is represented by some range in the interval $[0, 2\pi)$ over which $\theta$ is allowed to vary.    
      \end{itemize}
     \end{column}

    % --- Column right ---
    \begin{column}{0.4\textwidth} % 50% of the text width
        \centering % Center the image within this column
        \includegraphics[width=0.6\textwidth]{imgs/03/fig2-1a.png} 
        % \captionof{figure}{A description of the image.}
    \end{column}
  \end{columns}

\end{frame}

\begin{frame}

  \begin{columns}[c]
    % --- Column 1: Text ---
    \begin{column}{0.6\textwidth}
      \begin{itemize}
        \item Alternatively, the door can be viewed from above and regarded as a planar body, which has three degrees of freedom.
        \item The hinge joint then imposes two independent constraints, again leaving only one independent coordinate ($\theta$).
      \end{itemize}
      In both cases the joints constrain the motion of the rigid body, thus reducing the overall degrees of freedom.
      In this section we derive precisely such a formula, called Grübler's formula, for determining the number of degrees of freedom of planar and spatial robots.
     \end{column}

    % --- Column 2: Figure ---
    \begin{column}{0.4\textwidth} % 50% of the text width
        \centering % Center the image within this column
        \includegraphics[width=0.6\textwidth]{imgs/03/fig2-1a.png} 
        % \captionof{figure}{A description of the image.}
    \end{column}
  \end{columns}

\end{frame}

\section{2.2.1 Robot Joints}
\begin{frame}{2.2.1 Robot Joints}

  Figure 2.3 illustrates the basic joints found in typical robots.
  Every joint connects exactly two links; joints that simultaneously connect three or more links are not allowed. 

  \begin{center}
    \includegraphics[width=0.8\textwidth]{imgs/03/fig2-3.png} 
  \end{center}

\end{frame}

\begin{frame}

  The revolute joint (R), also called a hinge joint, allows rotational motion about the joint axis.
  
  \begin{center}
    \includegraphics[width=0.6\textwidth]{imgs/03/fig2-3R.png} 
  \end{center}

\end{frame}

\begin{frame}

  The prismatic joint (P), also called a sliding or linear joint, allows translational (or rectilinear) motion along the direction of the joint axis.

  \begin{center}
    \includegraphics[width=0.6\textwidth]{imgs/03/fig2-3P.png} 
  \end{center}

\end{frame}

\begin{frame}

  The helical joint (H), also called a screw joint, allows simultaneous rotation and translation about a screw axis.
  Revolute, prismatic, and helical joints all have one degree of freedom. 

  \begin{center}
    \includegraphics[width=0.6\textwidth]{imgs/03/fig2-3H.png} 
  \end{center}

\end{frame}

\begin{frame}

  Joints can also have multiple degrees of freedom.
  The cylindrical joint (C) has two degrees of freedom and allows independent translations and rotations about a single fixed joint axis.

  \begin{center}
    \includegraphics[width=0.6\textwidth]{imgs/03/fig2-3C.png} 
  \end{center}

\end{frame}

\begin{frame}

  The universal joint (U) is another two-degreeof-freedom joint that consists of a pair of revolute joints arranged so that their joint axes are orthogonal.

  \begin{center}
    \includegraphics[width=0.6\textwidth]{imgs/03/fig2-3U.png} 
  \end{center}

\end{frame}

\begin{frame}

  The spherical joint (S), also called a ball-and-socket joint, has three degrees of freedom and functions much like our shoulder joint.

  \begin{center}
    \includegraphics[width=0.6\textwidth]{imgs/03/fig2-3S.png} 
  \end{center}

\end{frame}

\begin{frame}

A joint can be viewed as providing freedoms to allow one rigid body to move relative to another.
It can also be viewed as providing constraints on the possible motions of the two rigid bodies it connects.
Generalizing, the number of degrees of freedom of a rigid body (three for planar bodies and six for spatial bodies) minus the number of constraints provided by a joint must equal the number of freedoms provided by that joint.

  \begin{center}
    \includegraphics[width=0.9\textwidth]{imgs/03/tab2_1.png} 
  \end{center}

\end{frame}

\section{2.2.2 Grübler's Formula}
\begin{frame}{2.2.2 Grübler's Formula}

  The number of degrees of freedom of a mechanism with links and joints can be calculated using Grübler's formula, which is an expression of Equation (2.3).
  
  \begin{center}
    \includegraphics[width=0.9\textwidth]{imgs/03/prop2-2.png}
  \end{center}

\end{frame}

\begin{frame}

  The number of degrees of freedom of a mechanism with links and joints can be calculated using Grübler's formula, which is an expression of Equation (2.3).
  
  \begin{center}
    \includegraphics[width=0.9\textwidth]{imgs/03/eq2-4.png} 
  \end{center}

\end{frame}

\begin{frame}{Example 2.4}

  Let us now apply Grübler's formula to several classical planar mechanisms.
  The k-link planar serial chain of revolute joints in Figure 2.5(a) below.
  \begin{columns}[T]
    % --- Column left ---
    \begin{column}{0.5\textwidth}
        \begin{center}
          \includegraphics[width=\textwidth]{imgs/03/fig2-5a.png}  
        \end{center}
    \end{column}

    % --- Column right ---
    \begin{column}{0.5\textwidth}
        $dof = m(N-1-J) + \sum\limits_{i=1}^{J}f_i$
    \end{column}
  \end{columns}
  
\end{frame}

\begin{frame}{Example 2.4}

  For the planar five-bar linkage of Figure 2.5(b) below.
  \begin{columns}[T]
    % --- Column left ---
    \begin{column}{0.5\textwidth}
        \begin{center}
          \includegraphics[width=\textwidth]{imgs/03/fig2-5b.png}  
        \end{center}
    \end{column}

    % --- Column right ---
    \begin{column}{0.5\textwidth}
        $dof = m(N-1-J) + \sum\limits_{i=1}^{J}f_i$
    \end{column}
  \end{columns}
  
\end{frame}

\begin{frame}{Example 2.3}

  (Four-bar linkage and slider-crank mechanism).
  The planar fourbar linkage shown in Figure 2.4(a) below consists of four links (one of them ground) arranged in a single closed loop and connected by four revolute joints.
  \begin{columns}[T]

    % --- Column left ---
    \begin{column}{0.5\textwidth}
        \begin{center}
          \includegraphics[width=\textwidth]{imgs/03/fig2-4a.png}  
        \end{center}
    \end{column}

    % --- Column right ---
    \begin{column}{0.5\textwidth}
        $dof = m(N-1-J) + \sum\limits_{i=1}^{J}f_i$
    \end{column}
  \end{columns}
  
\end{frame}

\begin{frame}{Example OMX}

  OpenManiplator-X

  \begin{columns}[T]
    % --- Column left ---
    \begin{column}{0.5\textwidth}
        \begin{center}
          \includegraphics[width=\textwidth]{imgs/03/OMX.png}  
        \end{center}
    \end{column}

    % --- Column right ---
    \begin{column}{0.5\textwidth}
        $dof = m(N-1-J) + \sum\limits_{i=1}^{J}f_i$
    \end{column}
  \end{columns}
  
\end{frame}

\begin{frame}{Example Lite6}

  UFactory Lite6
  
  \begin{columns}[T]
    % --- Column left ---
    \begin{column}{0.5\textwidth}
        \begin{center}
          \includegraphics[width=\textwidth]{imgs/03/lite6.png}  
        \end{center}
    \end{column}

    % --- Column right ---
    \begin{column}{0.5\textwidth}
        $dof = m(N-1-J) + \sum\limits_{i=1}^{J}f_i$
    \end{column}
  \end{columns}
  
\end{frame}

\begin{frame}{Example 2.6}

  (Redundant constraints and singularities). For the parallelogram linkage of Figure 2.7(a).
  
  \begin{columns}[T]
    % --- Column left ---
    \begin{column}{0.5\textwidth}
        \begin{center}
          \includegraphics[width=\textwidth]{imgs/03/fig2-7a.png}  
        \end{center}
    \end{column}

    % --- Column right ---
    \begin{column}{0.5\textwidth}
        $dof = m(N-1-J) + \sum\limits_{i=1}^{J}f_i$
    \end{column}
  \end{columns}
  
\end{frame}

\begin{frame}{Example 2.6}

  $dof = m(N-1-J) + \sum\limits_{i=1}^{J}f_i$

  $N = 5$, $J = 6$, and $f_i = 1$ for each joint.
  From Grübler's formula, the number of degrees of freedom is $3(5 - 1 - 6) + 6 = 0$.
  A mechanism with zero degrees of freedom is by definition a rigid structure.
  It is clear from examining the figure, though, that the mechanism can in fact move with one degree of freedom.
  Indeed, any one of the three parallel links, with its two joints, has no effect on the motion of the mechanism,
   so we should have calculated $dof = 3(4 - 1 - 4) + 4 = 1$.
  In other words, the constraints provided by the joints are not independent, as required by Grübler's formula.
  
  % \begin{center}
  %   \includegraphics[width=\textwidth]{imgs/03/fig2-7a.png}  
  % \end{center}
  
\end{frame}

\section*{} % Final slide use an unnumbered section
\begin{frame}{Questions?}

  Next:
  \begin{itemize}
    \item 04
  \end{itemize}

\end{frame}

\end{document}
