% Beamer presentation with Memorial University theme
\documentclass{beamer}

% Use custom Memorial University theme (explicit input to ensure findability)
% Minimal Memorial beamer theme: only red/grey/black/white
% Keeps styles minimal to produce a clean title page

\mode<presentation>

% Colors
\definecolor{MemorialRed}{RGB}{134,38,51}
\definecolor{MemorialGrey}{RGB}{99,102,106}
\definecolor{MemorialWhite}{RGB}{255,255,255}
\definecolor{MemorialBlack}{RGB}{35,31,32}

% Palette
\setbeamercolor{palette primary}{bg=MemorialRed, fg=MemorialWhite}
\setbeamercolor{palette secondary}{bg=MemorialGrey, fg=MemorialWhite}
\setbeamercolor{structure}{fg=MemorialRed}
\setbeamercolor{title}{fg=MemorialWhite, bg=MemorialRed}
\setbeamercolor{frametitle}{fg=MemorialWhite, bg=MemorialRed}
\setbeamercolor{normal text}{fg=MemorialBlack, bg=MemorialWhite}

% Fonts
\usefonttheme{professionalfonts}
\setbeamerfont{title}{size=\Large, series=\bfseries}
\setbeamerfont{frametitle}{size=\large, series=\bfseries}

\setbeamersize{text margin left=0.5cm, text margin right=0.5cm}

% Remove navigation symbols
\setbeamertemplate{navigation symbols}{}

% Headline with section name on the right
\setbeamertemplate{headline}{%
  \begin{beamercolorbox}[wd=1\paperwidth, ht=0.45cm, dp=0pt]{palette secondary}
    \vbox to 0.45cm{\vfil%
      \hbox to 1\paperwidth{%
        \hspace{1em}\hfill\usebeamerfont{section in head/foot}\insertsectionhead\hspace{1em}%
      }%
      \vfil}
  \end{beamercolorbox}
}

% Simple footline: muted background, left author, right page numbers
\setbeamertemplate{footline}{%
  \begin{beamercolorbox}[wd=1\paperwidth, ht=0.45cm, dp=0pt]{palette secondary}
    \vbox to 0.45cm{\vfil%
      \hbox to 1\paperwidth{%
        \hspace{1em}\usebeamerfont{author in head/foot}\insertshortauthor\hfill\insertframenumber/\inserttotalframenumber\hspace{1em}%
      }%
      \vfil}
  \end{beamercolorbox}
}

% Boxed title page (restores a visually framed title like the beamer default)
\setbeamertemplate{title page}{%
  \begin{center}
    \vspace{1cm}
    \begin{beamercolorbox}[sep=1.25em,center,rounded=true,shadow=true]{title}
      {\usebeamerfont{title}\inserttitle}\par\vskip0.5em
      {\usebeamerfont{subtitle}\insertsubtitle}
    \end{beamercolorbox}
    \vskip1.2em
    {\usebeamerfont{author}\insertauthor}\par\vskip0.5em
    {\usebeamerfont{date}\insertdate}
    \vfill
  \end{center}
}

\mode<all>



% Define email
\def\email{vpradodafons@mun.ca}

\title{Sample Beamer Slide}
% short author (used in footer) set to email
\author{Vinicius Prado da Fonseca, Ph.D. \\ vpradodafons@mun.ca}
\date{\today}

\begin{document}

\begin{frame}
  \titlepage
\end{frame}

\begin{frame}{Outline}

  \begin{columns}[T]
        \begin{column}{\textwidth}
          \tableofcontents
        \end{column}
  \end{columns}

\end{frame}

\section{First Section}
\begin{frame}{Example}

  This is a simple Beamer slide for testing `pdflatex`.

\end{frame}

\section{Second Section}
\begin{frame}

    \begin{columns}[T]
        % --- Column 1: Text ---
        \begin{column}{0.3\textwidth}
          \footnotesize
          \begin{itemize}
            \item Robot mechanisms can be arranged in a serial fashion like the open-chain arm in Figure 1.1
            \item Or forming closed loops, such as the Stewart-Gough platform.
            \item Open-chain, all the joints are actuated.
            \item Closed loops, only a subset of the joints may be actuated.
          \end{itemize}
        \end{column}

        % --- Column 2: Figure ---
        \begin{column}{0.7\textwidth} % 50% of the text width
            \centering % Center the image within this column
            \includegraphics[width=\linewidth]{imgs/01/fig1-1.png} 
            % \captionof{figure}{A description of the image.}
        \end{column}
    \end{columns}

\end{frame}

\section*{} % <--- Use an unnumbered section
\begin{frame}{Questions?}

  Next:
  \begin{itemize}
    \item 02 - Ch2 Intro, 2.1, 2.2
  \end{itemize}

\end{frame}

\end{document}
